\section{Introduzione}
\section{Controllo ON-OFF}
Si tratta di un tipo di controllo molto semplice che in cui gli
attuatori o si accendono o si spengono.
\section{Controllo proporzionale}
Il principio alla base di questo tipo di controllo è l'errore fra la
variabile misurata e il suo valore atteso; di conseguenza l'attuatore
viene regolato di modo da muovere il sistema vero il valore atteso.
\section{Controllo integrale}
\section{PID}
\section{Feed-Forward}
Quando si conoscono in anticipo le evoluzioni del sistem è possibile
agire sull'attuatore prima che si verifichino di modo da compensarne
gli effetti.
\section{Controllo mediante calcolatore}
Se variabili di controllo vengono campionate ed analizzate da un
calcolatore.

Mediante calcolatore è possibile implementare anche gli algoritmi di
controllo classici.