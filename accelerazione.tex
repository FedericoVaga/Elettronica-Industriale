\section{Accelerometro massa-molla}
Consiste in una scatola in cui è presente una piccola massa che può
muoversi orizzontalmente e che è soggetta ad un'azione di richiamo da
parte di una molla; nello stesso contenitore è presente un trasduttore
di posizione lineare che misura la posizione della massa. Indicando
con $x$ lo spostamento relativo della massa $m$ e applicando le leggi
della fisica per le molle ($k$ costante elastica della molla) possiamo
facilmente ricavare l'accelerazione $a$:

	\[F=ma=-kx \Rightarrow x=-\frac{ma}{k}\]

Data che c'è proporzionalità fra lo spostamento $x$ e l'accelerazione
$a$, possiamo definire la sensibilità del trasduttore come rapporto
fra lo spostamento e l'accelerazione:

	\[x=-\frac{ma}{k} \Rightarrow \frac{x}{a}=-\frac{m}{k}\]

\section{Servo-accelerometro}
Un servo-accelerometro è costituito da una massa $m$ collegata ad un
motore in continua $C$ di modo da far oscillare il rotore del motore.
Quando la massa subisce un'accelerazione questa si sposta e genera
una coppia proporzionale all'accelerazione. Un trasduttore di
posizione lineare rileva lo spostamento della massa rispetto alla sua
posizione di riposo; il segnale in uscita da questo trasduttore è
applicato ad un amplificatore la cui corrente in uscita è applicata
al motore che genera una coppia uguale e contraria a quella fornita
dalla massa. In questo modo la massa ritorna alla sua posizione di
riposto. La tensione misurata sul carico situalo fra il motore e
l'amplificatore ci darà una tensione proporzionale all'accelerazione.

\section{Accelerometro piezoelettrico}
Il principio alla base di un accelerometro piezoelettrico è lo stesso
del massa-molla con l'aggiunta di un cilindretto di materiale
piezoelettrico che assolve le funzioni di richiamo della molla e di
misurazione dello spostamento.
%FIXME trattazione fisica del piezo

\section{Accelerometro MEMS}
Gli accelerometri di questo tipo sono realizzati con tecnologia MEMS
(Micro-Electro-Mechanical-System). Questo tipo di tecnologia permette
di realizzare sensori, attuatori e altri dispositivi elettrinici
mediante un processo di lavorazione di uno strato di silicio.
Un accelerometro di tipo MEMS ha lo stesso principio del massa-molla
ma è realizzato mediante l'uso di gas. All'interno dell'accelerometro
viene creata un piccola bolla di gas riscaldato che si muove in
funzione degli spostamenti. Mediante dei sistemi di misurazione della
temperatura, distribuiti dentro all'accelerometro, è possibile
individuare lo spostamento della bolla quindi ricavare l'accelerazione
che ne ha generato lo spostamento.