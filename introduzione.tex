Lo scopo di un trasduttore è quello di trasformare un segnale di
natura fisica in un segnale elettrico; per fare questo i trasduttori
si basano su principi fisici.
Per descrivere le caratteristiche di un trasduttore possiamo
catalogarle in statiche (cioè proprie del trasduttore) e dinamiche
(cioè che dipendono dal processo al quale lo si applica).
Fra le caratteristiche statiche abbiamo:
\begin{itemize}
 \item \textit{Sensibilità}: rapporto fra la variazione di grandezza
elettrica e di grandezza fisica.
	\[ \frac{\Delta GE}{\Delta GF} \]
 \item \textit{Risoluzione}: cioè la minima variazione valutabile
della grandezza fisica. Di conseguenza avremo una rappresentazione
della grandezza fisica con un segnale a gradini. Questa è la diretta
conseguenza del passaggio dallo spazio continuo (analogico/fisico)
allo spazio discreto (digitale).
 \item \textit{Soglia}: il minimo valore della grandezza fisica
misurabile.
 \item \textit{Range}: intervallo di valori della grandezza fisica in
esame che il trasduttore può misurare.
 \item \textit{Isteresi}: massima differenza tra due cammini di
andata e di ritorno dell'uscita del trasduttore durante un ciclo che
raggiunge gli estremi del range.
 \item \textit{condizioni ambientali}: condizioni ambientali che
garantiscono il corretto funzionamento; se il trasduttore viene fatto
lavorare fuori dalle condizioni nominali si possono generare errori.
 \item \textit{errore}: differenza fra il comportamento reale e
quello ideale, a questo viene anche associata una
 \item \textit{banda d'errore}: è una regione in cui i valori reali
possono differire da quelli ideali.
\end{itemize}

Le caratteristiche dinamiche di un trasduttore dipendono dal
funzionamento del processo al quale viene applicato, il problema
principale è verificare se il trasduttore scelto è in grado di seguire
fedelmente le variazioni del processo; per capirlo vengono forniti i
diagrammi di Bode della sensibilità del trasduttore (fig. d) e nel
caso in cui il trasduttore non riesca ad operare a frequenze troppo
basse viene fornita la banda di applicabilità (fig. e). Altra
caratteristica dinamica importante è l'affidabilità del trasduttore,
cioè la capacità di sopportare sovraccarichi, la vita media del
trasduttore.