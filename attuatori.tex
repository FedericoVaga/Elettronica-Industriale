\section{Motori in corrente continua}
In questo tipo di motori, la coppia applicata all'albero è
proporzionale alla corrente circolante negli avvolgimenti del motore
$\Gamma = KI$. Inoltre la tensione ai capi del motore risulta
proporzionale alla velocità del motore; questo vale sotto l'ipotesi di
rendimento massimo, ovvero senza perdite di energia, ne consegue che
$V=K\omega$. Da queste considerazioni ne consegue che
$\Gamma\omega=VI$

\section{Motori passo-passo}
I motori passo-passo sono motori comandati da segnali logici. Questo
ha il vantaggio della perfetta riproduzione dei movimenti e della
capacità di mantenere la posizione raggiunta. Il motore si basa fra
l'accoppiamento di un ingranaggio di materiale ferromagnetico e di 4
avvolgimenti statorici contigui. Gli avvolgimenti vengono posti ad
una distanza di un angolo pari ad $\frac{1}{4}$ dell'angolo fra due
denti dell'ingranaggio.

Gli avvolgimenti sono alimentati due alla volta in ordine
sequenziale, questo muove l'ingranaggio. Il dente dell'ingranaggio
allineato al primo avvolgimento viene attratto dai successivi man
mano che questi vengono alimentati, questo produce la rotazione
dell'ingranaggio. Al termine della rotazione, il dente successivo
viene attratto dagli avvolgimenti.

\section{Attuatore elettromagnetico}
Questo tipo di attuatore è realizzato come il trasduttore di velocità
lineare. Una barretta magnetizzata è posta parzialmente all'intero di
un solenoide. Nel momento in cui il solenoide viene alimentato, la
berretta subisce una forza di attrazione o repulsione dal solenoide
che è proporzionale alla corrente che scorre nel solenoide. Sia $v$ la
velocità con cui si sposta il magnete, $n$ il numero di spire, $V$ la
tensione ai capi del solenoide e $I$ la corrente che vi scorre;
possiamo calcolare la forza applicata:

	\[V=n\phi_Bv, VI=Fv\]
	\[F= \frac{VI}{v}=n\phi_B I\]